\chapter*{Abstract\markboth{Abstract}{Abstract}}
Blockchain allows to have a distributed peer-to-peer network where non- trusting members can interact with each other without a trusted intermediary, in a verifiable manner. Blockchain is currently being used to implement peer-to-peer electronic cash systems, optimize supply chain
management, keep track of land records digitally and provide digital degrees. The current insurance system in India involves the insurance providers, health centers and clients to deal with intermediaries and third parties such as reinsurers, insurance agents and credit monitoring agencies for customer verification, policy servicing and claim settlement. The system in its current form depends on these intermediaries for transparency in the entire life cycle of insurance process. The entire process is complex, tedious and archaic and is also not cost-efficient for the insurance providers as they have to employ multiple intermediaries. Premium facilities like quick claim settlements, cashless insurance and on-site KYC are provided only through tie-ups between insurance providers and top tier health centers thus limiting access to such facilities to a certain class of society.  Blockchain helps in streamlining the existing insurance system and makes it accessible to people from all walks of life. A decentralized insurance platform where the healthcare centers, insurance providers and clients participate in a trustless, peer-to-peer network where the medical records and policy details of the clients are encoded and stored on the blockchain. These records are accessed only by the parties involved in the policy- servicing agreement. KYC, policy-servicing and claim settlement can be handled in a quick and efficient manner through smart contracts. All transactions in the system are verified using Proof-of-Authority (PoA) protocol ensuring transparency in the system. This can also help in
reducing fraud related to the integrity of a policy or claim. Blockchain will minimize counterfeiting, double booking, document or contract alterations. However, use of the Blockchain does not mitigate the risk associated with the majority of first party and third-party frauds.

\textit{Keywords: Blockchain, distributed peer-to-peer network, trustless, decentralized, smart contract, Proof-of-Authority (PoA).
}

